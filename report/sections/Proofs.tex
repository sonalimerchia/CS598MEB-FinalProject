% chktex-file 8
\section{Proofs}

ZCNT is proposed as a distance metric~\cite{zcnt_paper}. In order to be a distance metric, ZCNT must satisfy the four following properties: 

\begin{enumerate}
    \item {\bf Identity:} $\zcnt (u, u) = 0$
    \item {\bf Positivity:} if $u \neq v$ then $0 < \zcnt (u, v)$
    \item {\bf Symmetry:} $\zcnt (u, v) = \zcnt (v, u)$
    \item {\bf Triangularity:} $\zcnt (u, v) \leq \zcnt (u, w) + \zcnt (w, v)$
\end{enumerate}

Below we will prove each of these properties mathematically. 

\subsection{Identity}

The identity property can be mathematically obtained without any additional lemmas. 

\begin{theorem}
    Let $u$ be some arbitrary copy-number profile under the ZCNT model. 
    
    Then $\zcnt (u, u) = 0$
\end{theorem}

\noindent {\bf Proof:}
\begin{align*}
    \textbf{Statement} & & \textbf{Explanation} \\
    \cline{1-3}
    \zcnt (u, u) &= \frac{1}{2} || \Delta(u) - \Delta(u) ||_1  & \text{Equation~\ref{eq:1}}\\ 
    &= \frac{1}{2} || \mathbf 0 ||_1 &\text{Subtract}\\ 
    &= \frac{1}{2} (0) &\text{Simplify} \\ 
    \zcnt (u, u) &= 0 &\text{Simplify}
\end{align*}

\subsection{Positivity}

\begin{theorem}
    Let $u$ and $v$ be some arbitrary copy-number profiles under the ZCNT model. 
    
    Then $u \neq v \implies 0 < \zcnt (u, v)$.
\end{theorem}


\noindent {\bf Proof:}

The proof of positivity is based on the principle that the L1 norm is a distance metric. Therefore by the positivity property, the L1 norm of any nonzero vector will be greater than $0$. We can write this formally as:
\begin{lemma}\label{lemma:l1_positivity}
    Let $u$ and $v$ be some arbitrary vector in $\mathbb{R}^n$. 
    
    Then $u \neq v \implies 0 < ||u - v||_1$.
\end{lemma}

Since $\Delta$ is a bijective function~\cite{zcnt_paper}, we know that every input has a unique output. We also know that every output has a unique input. Therefore if $u \neq v$, then the delta mappings will be different. We can write this formally as: 

\begin{lemma}\label{lemma:delta_bijection}
    Let $u$ and $v$ be some copy number profiles under the ZCNT model. Let $\Delta(\cdot)$ be the function to convert a copy number profile to its respective delta mapping. 
    
    Then $u \neq v \iff \Delta(u) \neq \Delta(v)$
\end{lemma}

Using these properties, we can write the following proof for the positivity property of ZCNT

\begin{align*}
    \textbf{Statement} & & \textbf{Explanation} \\
    \cline{1-3} \\
    u \neq v &\implies \Delta(u) \neq \Delta(v) & \text{Lemma}~\ref{lemma:delta_bijection} \\ 
    & \implies \Delta(u) - \Delta(v) \neq 0 & \text{Subtraction}\\
    & \implies 0 < ||\Delta(u) - \Delta(v)||_1 & \text{Lemma}~\ref{lemma:l1_positivity} \\
    & \implies 0 < \frac{1}{2}||\Delta(u) - \Delta(v)||_1 & \text{Multiply} \\
    u \neq v & \implies 0 < \zcnt (u, v) & \text{Equation}~\ref{eq:1}
\end{align*}

\subsection{Symmetry}
\begin{theorem}
    Let $u$ and $v$ be some arbitrary copy-number profiles under the ZCNT model. 
    
    Then $\zcnt (u, v) = \zcnt (v, u)$
\end{theorem}

\noindent {\bf Proof:}

\begin{corollary}\label{corollary:abs_val}
    $\forall x \in \mathbb{R}$, $|x| = |-x|$ by absolute value definition
\end{corollary}

\begin{align*}
    \textbf{Statement} & & \textbf{Explanation} \\
    \cline{1-3} \\
    \zcnt (u, v) &= \frac{1}{2} || \Delta(u) - \Delta(v) ||_1 & \text{Equation}~\ref{eq:1} \\ 
    &= \frac{1}{2} || -(\Delta(u) - \Delta(v)) ||_1 & \text{Corollary}~\ref{corollary:abs_val} \\
    &= \frac{1}{2} ||  \Delta(v) - \Delta(u) ||_1 & \text{Multiply}\\ 
    \zcnt (u, v) &= \zcnt(v, u) & \text{Equation}~\ref{eq:1}
\end{align*}

\subsection{Triangularity} 

\begin{theorem}
    Let $u$ and $v$ be some arbitrary copy number profiles under the ZCNT model. 

    Then for all copy number profiles $w$, 
    
    $\zcnt (u, v) \leq \zcnt (u, w) + \zcnt (w, v)$
\end{theorem}

\noindent {\bf Proof:}

Let $\mathbf x = (\Delta(u) - \Delta(w))$ and $\mathbf y = (\Delta(w) - \Delta(v))$. From this we get: 

\begin{alignb}\label{eq:substitutions}
    ||\mathbf{x}||_1 &= ||\Delta(u) - \Delta(w)||_1 \\ 
    &=2 \times\zcnt(u, w) \\
    ||\mathbf{y}||_1 &= ||\Delta(w) - \Delta(v)||_1 \\ 
    &=2 \times\zcnt(w, v) \\
    ||\mathbf{x} + \mathbf{y}||_1 &= ||\Delta(u) - \Delta(w) + \Delta(w) - \Delta(v)||_1 \\ 
    &= ||\Delta(u) - \Delta(v)||_1 \\ 
    &= 2 \times\zcnt(u, v)
\end{alignb} 

Since L1 is a distance metric we can write its triangularity property formally as follows: 

\begin{lemma}\label{lemma:l_triangular}
    $\forall x, y \in \mathbb{R}^n, ||\mathbf x + \mathbf y||_1 \leq ||\mathbf x||_1 + ||\mathbf y||_1$
\end{lemma}

Using these two equations, we can formally prove the triangularity property of ZCNT as follows: 

\begin{align*}
    \textbf{Statement} & & \textbf{Explanation} \\
    \cline{1-3} \\
    ||\mathbf x + \mathbf y||_1 & \leq ||\mathbf x||_1 + ||\mathbf y||_1 & \text{Lemma}~\ref{lemma:l_triangular} \\ 
    2 \times \zcnt(u, v) & \leq 2[\zcnt(u, w) + \zcnt(w, v)] & \text{Substitutions~\ref{eq:substitutions}} \\ 
    \zcnt(u, v) &\leq \zcnt(u, w) + \zcnt(w, v) & \text{Divide}
\end{align*}