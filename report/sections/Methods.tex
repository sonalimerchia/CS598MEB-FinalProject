\section{Methods}
Since ZCNT is a metric that is meant to emulate CNT, the question of behavior across symmetry exists. CNT is not a symmetric property and doesn't always exist. If ZCNT typically approximates CNT, how does it behave when the corresponding CNT distance does not exist? Does it trend towards $+\infty$ when $\cnt(u, v) = \infty$ or is there another behavior?

\vspace{10pt}

\noindent The points of interest in this paper are: 

\begin{enumerate}
    \item How well does ZCNT replicate CNT\@? How well does it replicate CN3?
    \item How does ZCNT behave when the corresponding CNT does not exist?
    \item How often do reachability issues show up in simulated data? 
    \item How often do reachability issues show up in real data? 
    \item Does ZCNT more closesly mimic median distance or mean correction?
\end{enumerate}

\subsection{Comparing ZCNT and CNT}\label{section:comp_dist}

In order to compare ZCNT and CNT distances, pairwise distances were generated for simulated data. The data was the same simulated data from the original ZCNT paper~\cite{zcnt_paper}. There were three kinds of distances generated for every simulation pair: ZCNT, CNT, and CN3. 

The distances were filtered so we only considered the distances where CNT for that pairing was not $+\infty$. Since both ZCNT and CN3 always exist at a noninfinite value, this was the only filter. Then the relative error was calculated for each distance pairing (all combinations of ZCNT, CNT, and CN3). 

\subsection{Behavior of ZCNT when CNT is infinite}\label{section:behavior}

In order to determine ZCNT behavior when CNT is infinite, the same simulated distances in Section~\ref{section:comp_dist} were used to classify copy-number pairings into three classes: 


{\bf Class 1:} $\cnt(u, v)$ and $\cnt(v, u)$ both exist

{\bf Class 2:} $\cnt(u, v)$ exists but $\cnt(v, u)$ does not 

{\bf Class 3:} Neither $\cnt(u, v)$ nor $\cnt(v, u)$ exist 

Then an analysis was done based on the distribution of the corresponding ZCNT distances. 


\subsection{Reachability in Simulated Data}\label{section:simulated_reachability}
The same simulated data from the prior two sections was put through Lazac, a ZCNT Large Parsimony algorithm~\cite{zcnt_paper}. Then, using the calculated pairwise distances in CNT and ZCNT, two statistics were taken: 

\begin{enumerate}
    \item The percentage of edges in the overall tree represent biologically infeasible transformations.
    \item The percentage of ancestor-descendant relationships that were biologically infeasible.
\end{enumerate}

We define ``biologically infeasible'' transformations to be transformations wherein a copy-number goes from zero to a positive number. These can easily be determined by calculating the CNT distances. If the CNT distance between two nodes is $\infty$, the transformation is biologically infeasible. 

For ancestor-descendant relationships between an ancestor $u$ and a descendant $v$, the pairing is considered illegal if and only if an edge drawn straight between them would be biologically infeasible.